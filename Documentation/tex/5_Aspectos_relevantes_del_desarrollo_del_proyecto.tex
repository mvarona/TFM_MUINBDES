\capitulo{5}{Aspectos relevantes del desarrollo del proyecto}

En este apartado se recogen los hechos más significativos del desarrollo del proyecto.

Como se ha comentado en el capítulo anterior, la metodología de trabajo elegida ha sido CRISP-DM, que se ha usado como hilo conductor del trabajo. A continuación, se procede a comentar lo más representativo de cada etapa, junto con el trabajo que se ha llevado a cabo en cada una:

\section{Entendimiento de negocio}

Como parte de esta etapa, se definieron los ocho objetivos detallados en la segunda sección de esta memoria, y se definieron los siguientes requisitos para poder alcanzar dichos objetivos:

\begin{enumerate}
    \item El sistema deberá recoger las preferencias del usuario, en el sentido de sus gustos personales, para poder mostrar municipios que las satisfagan.
    \item Estas preferencias se expresarán a través de un formulario de diversas preguntas, que permitan segmentar los municipios candidatos hasta encontrar los que reúnan las características deseadas.
    \item Se implementará un sistema de recomendación basado en contenido, para lo que se deberá recoger la entrada del usuario explícitamente. Adicionalmente, se podrá recoger la entrada implícitamente en otros casos de uso.
    \item Se recogerá también el gusto o no de los usuarios por el resultado ofrecido, de manera que se puedan tener datos para construir un sistema de recomendación basado en filtro colaborativo.
    \item Se explorará un sistema de recomendación basado en filtro colaborativo.
    \item Se recopilarán datos significativos basándose en otros trabajos relacionados y en los que el autor considere relevantes, que permitan crear perfiles detallados para cada municipio, de entre los conjuntos y extracciones de datos disponibles para el ámbito del proyecto. En particular, siempre que sea posible se incluirán datos de: nombre completo del municipio, provincia y comunidad autónoma a las que pertenece, número de habitantes, superficie, densidad de población, disponibilidad de servicios educativos (colegios y universidades), disponibilidad de servicios sanitarios (centros de salud y hospitales), datos climáticos históricos básicos, datos de renta y empleo básicos, datos de venta y alquiler de viviendas en el municipio y en su provincia, cercanía a la capital de provincia, cercanía a la costa y a la montaña, altitud, datos de cobertura de Internet en el municipio o en su provincia, principales servicios de los que dispone, ubicación y extractos de información textual y fotográfica.
    \item El sistema será capaz de arrojar recomendaciones para un 70 \% de los casos probados, y su rendimiento se medirá por métricas ajustadas a la naturaleza de su modelo, buscando alcanzar una fiabilidad adecuada.
    \item Se diseñará el sistema de forma escalable; es decir, se considerará desde su diseño en posibles repeticiones automáticas del proceso de extracción, transformación y limpieza de datos, posibles adiciones o supresiones de la base de datos de municipios españoles, posibles incorporaciones, modificaciones o supresiones de datos para uno o varios municipios, posibles traducciones o posibles casos de uso similares a los existentes y potencialmente deseables.
    \item Se asignará responsabilidad única a los componentes informáticos que formen el trabajo.
    \item Se presentará un resumen de los datos de cada municipio al usuario.
    \item Se creará un sistema versionado, de fácil despliegue, actualización e interacción.
    \item Se creará un sistema con la privacidad y la seguridad en mente, que no expondrá secretos ni credenciales, que aplicará las mejores prácticas estándar para el ámbito del proyecto, que no recogerá datos personales y que contará con sistemas de contacto para comunicarse con el responsable del sistema.
    \item Se realizará una evaluación de la utilidad, usabilidad y rendimiento del sistema por parte de usuarios potenciales reales, a fin de detectar puntos de mejora y evaluar las decisiones de diseño e implementación adoptadas.
\end{enumerate}

Como criterios de éxito, además del valor de capacidad para obtener resultados establecido anteriormente, se requerirá una evaluación positiva, de al menos un 70 \% de satisfacción y un 50 \% de expectativa de uso por parte de los usuarios potenciales con los que se probará el sistema.

Para desarrollar el sistema se contará con dos ordenadores a disposición del alumno, el servidor personal mencionado anteriormente –con pila de tecnologías LAMP: Linux, Apache, MySQL y PHP–, y la infraestructura propia de Google –con entorno preparado para ejecutar aplicaciones Web de Python– donde se desplegará el resultado final del proyecto.

El desarrollo del trabajo tiene una duración estimada de alrededor de dos meses, y para completarlo se realizará un análisis de viabilidad previa, recolección y adecuación de los datos necesarios, integración de los datos y composición del modelo, evaluación del modelo e interpretación, presentación y despliegue del resultado final.

\section{Entendimiento de datos}

Una vez completado el análisis de viabilidad con éxito, gracias al descubrimiento de trabajos relativamente relacionados y de los conocimientos del autor en aplicaciones Web y su compatibilidad con otros sistemas basados en Python, se procedió a llevar a cabo la recolección y exploración inicial de datos. Para explicarla, se procede a detallar la función y contenido de cada componente software usado en el proceso de extracción. Nótese que, por fines de compartición de conocimiento y adecuación a los estándares del mundo académico y profesional, los nombres de archivos, variables y otros literales ajenos al modelo se expresan en inglés; mientras que los nombres de las columnas o variables propias de las fuentes de datos mantienen su nombre original en español. Por otra parte, en el caso de nombres escritos en idiomas co-oficiales se ha procurado mantener la convención del Instituto Nacional de Estadística, o la usada mayoritariamente entre las fuentes, en caso de conflicto entre varios nombres alternativos procedentes de diversas fuentes. Es importante recordar que se parte de la base de datos de municipios del INE en formato XLSX, que ha sido convertida a CSV y cuenta con las siguientes columnas: código numérico del municipio, nombre del municipio y provincia a la que pertenece. 

A continuación, se resumen los aspectos más relevantes de esta fase:

\begin{enumerate}
    \item Se hace uso de archivos no añadidos al repositorio público para contener las variables de entorno con secretos y credenciales de uso en los servicios de terceros usados.
    \item El cálculo de distancias a la costa y a la montaña desde cada municipio se ha realizado mediante heurísticas basadas en la presencia de costa en cada provincia, y en la importancia de sus sistemas montañosos, conforme a lo explicado anteriormente.
    \item Para los municipios cuya renta anual no es pública, bien por estar agrupada y promediada debido a su tamaño (menos de 1000 habitantes), o por secreto estadístico, se han tomado los valores oficiales más cercanos; esto es, el valor publicado de la agrupación para dichos municipios en esa provincia, y el valor publicado para el municipio más cercano, respectivamente.
    \item Se ha trabajado con el código del municipio asignado por el INE siempre que ha sido posible; sin embargo, debido a su baja adopción, ha sido necesario trabajar con dos claves primarias: municipio y provincia. Posteriormente, los datos de cada fuente que no lo tuvieran se han fusionado con el código del INE para poder realizar búsquedas unívocas y homogéneas.
    \item Dado que la mayoría de proveedores, de titularidad pública y privada, trabajan con el nombre del municipio y su provincia, es necesario tener en cuenta la frecuente normalización del nombre del municipio; es decir, la omisión de determinantes o su puesta al final del nombre, así como el predominio de la variante del idioma co-oficial local en la mayoría de los casos.
    \item Se emplean los datos de diversos terceros para obtener, por ejemplo, la geolocalización de cada municipio, su elevación, sus principales características o su tiempo atmosférico promedio en febrero y en julio. En particular, y como se explica detalladamente en el Anexo \ref{programador} \guillemotleft Documentación técnica de programación\guillemotright, en el caso de la localización de los municipios, se detectaron algunas inconsistencias en la fase de exploración de los datos; dado que el servicio localizó municipios homónimos fuera de España. Tras incluir el país como restricción para el servicio, la exploración de datos de este conjunto fue según lo esperado.
    \item Se ha realizado extracción de datos Web, o \textit{Web scraping}, para diversos conjuntos de datos, como los centros de educación superior o los datos inmobiliarios por municipio. Como se explica en el mencionado Anexo, esta extracción automatizada presentó diversos retos; como la necesidad de esperar un pequeño tiempo (uno o dos segundos) entre cada petición para no sobrecargar el servidor de destino, o la combinación de varias páginas o proveedores para obtener el resultado final.
    \item Se han creado diversos ficheros con funciones auxiliares, como fusionar conjuntos de datos hasta crear el modelo, extraer los nombres co-oficiales locales de los municipios, o eliminar columnas de diversos subconjuntos.
    \item El cálculo de la distancia por carretera entre cada municipio y la capital de su provincia se ha realizado mediante la herramienta OpenTripPlanner (OTP)~\cite{otp}. Básicamente, suministrándole el mapa de España, esta herramienta de software libre crea un servidor al que es posible realizar consultas de creación de rutas mediante una interfaz gráfica Web y una API HTTP. Este servidor se desplegó en un ordenador de más memoria RAM que el que se ha usado para desarrollar el trabajo, ya que el mapa requería al menos 28 GigaBytes, pero en la misma red doméstica que el principal del alumno. De esta manera, los archivos que realizan la extracción se conectan a través de la API HTTP al servicio, obteniendo los kilómetros que separan cada localidad de su capital de provincia.
    \item Se ha obtenido una lista de los veinte lugares más significativos de cada municipio, a través de la API del proveedor GeoApify~\cite{geoapify}, que presentaba la tarificación más atractiva para este propósito. Una vez extraídos, se realizan cálculos para ver qué categorías de las siguientes son las más repetidas: comercios, turismo, alojamientos, ocio, naturaleza, servicios, actividades, entretenimiento, hostelería y deporte, además de otras devueltas adicionalmente por el servicio. Posteriormente, se incorporan estos datos al modelo para tener información cualitativa de cada municipio.
    \item Los datos climáticos recogidos a través de la API de OpenWeather~\cite{openweather} son los siguientes:

    \begin{itemize}
        \item Promedio de las temperaturas mínimas en febrero.
        \item Promedio de las temperaturas máximas en febrero.
        \item Promedio de las temperaturas en febrero.
        \item Promedio de humedad en febrero.
        \item Promedio de la velocidad media del viento en febrero.
        \item Promedio de las precipitaciones mínimas en febrero.
        \item Promedio de las precipitaciones máximas en febrero.
        \item Promedio de las precipitaciones en febrero.
        \item Promedio de porcentaje de nubosidad en febrero.
        \item Promedio de número de horas de sol en febrero.
    
        \item Promedio de las temperaturas mínimas en julio.
        \item Promedio de las temperaturas máximas en julio.
        \item Promedio de las temperaturas en julio.
        \item Promedio de humedad en julio.
        \item Promedio de la velocidad media del viento en juliio.
        \item Promedio de las precipitaciones mínimas en julio.
        \item Promedio de las precipitaciones máximas en julio.
        \item Promedio de las precipitaciones en julio.
        \item Promedio de porcentaje de nubosidad en julio.
        \item Promedio de número de horas de sol en julio.
    \end{itemize}

    \item A través de la API de Wikipedia se ha extraído una breve descripción del municipio, así como algunas de las principales imágenes de su entrada en dicha enciclopedia. Estos textos e imágenes se usarán en el resultado final con fines ilustrativos, para que el usuario pueda conocer mejor el municipio recomendado y pueda aprender más sobre él. Con el fin de obtener imágenes de la localidad y su entorno, se excluyen arbitrariamente imágenes cuyo nombre involucre mapas, escudos, banderas, planos, etc.
    
\end{enumerate}

Una descripción exhaustiva de cada archivo involucrado en esta fase, y una explicación más detallada de los aspectos anteriores pueden encontrarse en el Anexo \ref{programador} \guillemotleft Documentación técnica de programación\guillemotright. 

Todos los datos han sido recogidos en diciembre de 2022, y hacen referencia a los datos más recientes disponibles a esa fecha, siendo las fechas más distintas a la fecha de recogida las de los datos:

\begin{itemize}
    \item Rentas brutas medias por municipio, pertenecientes al año 2019.
    \item Cobertura de banda ancha por municipio, perteneciente al año 2020.
\end{itemize}

El perfil estadístico de los datos que compondrán el modelo puede encontrarse en el Anexo \ref{diseno} \guillemotleft Especificación de diseño\guillemotright.

\section{Preparación de datos}

Una vez la exploración de los datos cuantitativos, que alimentarán posteriormente el modelo, fue exitosa en todos los conjuntos de datos, dio comienzo la fase de preparación de los datos.

En esta fase se eliminaron los marcadores de datos nulos o no encontrados (establecidos arbitrariamente como \guillemotleft -\guillemotright\space), y se sustituyeron por campos vacíos.

\imagen{modelo1}{Datos geográficos y de lugares de interés de los municipios antes de su transformación.}

Además, se convirtió la expresión de los lugares más frecuentes para un municipio de cualitativa, para la exploración de datos, a cuantitativa siguiendo la codificación \textit{one hot encoding}.

\imagen{modelo2}{Datos geográficos y de lugares de interés de los municipios después de su transformación.}

Antes de continuar, se creó una versión del modelo hasta el momento en formato JSON, mucho más conveniente a la hora de recuperar los datos de un municipio en particular para presentarlos al usuario, tanto por razones de comodidad como de rendimiento. Así, el sistema de recomendación devolverá el código INE del municipio, y sus datos se recuperarán de este archivo que, por este motivo, estará formado por un diccionario indexado por estos códigos.

Posteriormente, se eliminaron las variables cualitativas (nombre codificado del municipio, nombre tradicional del municipio y comunidad autónoma), y las coordenadas (campos de latitud y longitud) de cada municipio, y se normalizaron los datos para disponer de ellos de forma agnóstica a las unidades empleadas en cada dimensión. La normalización se hizo basándose en diversas variables, como la renta bruta media o la tasa de empleo; sin embargo, la que mejores resultados arrojó fue el tamaño de la población. Así, se utilizó la regresión de Ridge –que reduce los coeficientes introduciendo un término de penalización igual a la suma de los coeficientes cuadrados por un coeficiente de penalización– para normalizar las variables sobre la clase, y se multiplicaron los coeficientes devueltos por la estructura de datos.

Para todo ello se usan los archivos \texttt{normalize.py} y \texttt{create-model.py}.

\section{Modelado}

A la hora de hablar del modelo del trabajo, es necesario distinguir entre los dos enfoques que se han explorado:

\subsubsection{Sistema de recomendación basado en contenido}

Una vez se dispone del modelo normalizado, este puede procesarse para producir el sistema de recomendación basado en contenido. Aquí se ha optado por basarlo en el cálculo de la similitud coseno entre el municipio que constituirá la entrada y los restantes, con el objetivo de devolver el más cercano del espacio vectorial que forman los municipios representados por vectores según sus características.

Tras calcular la similitud coseno se ordenarán los valores de forma descendente, para seleccionar a continuación el resultado de mayor similitud, que será el municipio presentado al usuario. Para ello, el código del municipio devuelto será usado para acceder a la base de datos de los municipios en formato JSON.

\imagen{cosine}{Explicación de cómo la similitud coseno calcula la distancia entre vectores, devolviendo valores más altos cuanto más similares son; y más bajos cuanto más opuestos son~\cite{US3}.}

Así, se definió una función que toma un código INE como entrada y tras leer el CSV del modelo calcula la similitud coseno entre ese municipio y el resto, devolviendo el código del INE correspondiente al primer resultado. En la fase de despliegue, se preparó el sistema para recuperar la información del municipio a partir de su código INE y devolverla al usuario.

\subsubsection{Sistema de recomendación de filtro colaborativo basado en usuarios}

Además del sistema de recomendación basado en contenido, el autor deseaba construir un sistema de recomendación basado en un filtro colaborativo que se alimentara de las valoraciones (positivas o negativas de forma binaria, \guillemotleft me gusta / no me gusta\guillemotright\space) de los usuarios al recibir la recomendación de un municipio. La idea detrás de estos sistemas es recomendar elementos que han gustado a usuarios similares. Debido a la baja tasa de valoración habitual, suelen trabajar con matrices dispersas y necesitar grandes cantidades de valoraciones de usuarios para poder realizar recomendaciones adecuadas.

Dado que el dominio del problema no dispone de conjuntos de datos con valoraciones de usuarios y, el propio sistema, aunque diseñado para recogerlas, no ha podido coleccionar las suficientes durante su creación por los limitados recursos de un trabajo académico como este, el autor consideró crear su propio conjunto de datos ficticio imitando usuarios reales.

Para ello, creó 10 usuarios ficticios y valoró con cada uno a 100 municipios de forma positiva y a 100 municipios de forma negativa. Estos municipios estuvieron elegidos aleatoriamente aplicando los siguientes criterios para cada usuario, con la intención de crear perfiles de usuarios intuitivamente naturales:

\begin{itemize}
    \item El primer usuario muestra su valoración positiva por 100 municipios costeros aleatorios, y su valoración negativa por 100 municipios muy alejados de la costa aleatorios; es decir, representa a un usuario que prefiere los municipios costeros.

    \item El segundo usuario muestra su valoración positiva por 100 municipios muy alejados de la costa aleatorios, y su valoración negativa por 100 municipios costeros aleatorios; es decir, representa a un usuario que prefiere los municipios alejados de la costa.

    \item El tercer usuario muestra su valoración positiva por 100 municipios cercanos a la montaña aleatorios, y su valoración negativa por 100 municipios muy alejados de la montaña aleatorios; es decir, representa a un usuario que prefiere la montaña.
    
    \item El cuarto usuario muestra su valoración positiva por 100 municipios muy alejados de la montaña aleatorios, y su valoración negativa por 100 municipios de montaña aleatorios; es decir, representa a un usuario que prefiere los municipios alejados de la montaña.

    \item El quinto usuario muestra su valoración positiva por 100 municipios de menos de 10 000 habitantes aleatorios, y su valoración negativa por 100 municipios de más de 10 000 habitantes aleatorios; es decir, representa a un usuario que prefiere los pueblos a las ciudades.

    \item El sexto usuario muestra su valoración positiva por 100 municipios de más de 100 000 habitantes aleatorios, y su valoración negativa por 100 municipios de menos de 10 000 habitantes aleatorios; es decir, representa a un usuario que prefiere las ciudades a los pueblos.

    \item El séptimo usuario muestra su valoración positiva por 100 municipios con lugares naturales aleatorios, y su valoración negativa por 100 municipios sin lugares naturales aleatorios; es decir, representa a un usuario que prefiere la naturaleza.

    \item El octavo usuario muestra su valoración positiva por 100 municipios con lugares de comercios aleatorios, y su valoración negativa por 100 municipios sin lugares de comercio aleatorios; es decir, representa a un usuario que prefiere las zonas comerciales.

    \item El noveno usuario muestra su valoración positiva por 100 municipios con menos de un 50 \% de conectividad de fibra óptica a más de 100 megabytes aleatorios, y su valoración negativa por 100 municipios con más de un 50 \% de dicha conectividad aleatorios; es decir, representa a un usuario que prefiere zonas poco conectadas a Internet.

    \item El décimo usuario muestra su valoración positiva por 100 municipios con más de un 50 \% de conectividad de fibra óptica a más de 100 megabytes aleatorios, y su valoración negativa por 100 municipios con menos de un 50 \% de dicha conectividad aleatorios; es decir, representa a un usuario que prefiere zonas conectadas a Internet.

\end{itemize}

Este proceso se realizó a través del \textit{script} \texttt{create\_ratings.py}. A continuación, una vez creados los usuarios anteriores y las 2000 valoraciones iniciales, se usó la librería \texttt{surprise}, de scikit-learn, para tratar con ellos. Tras especificarle la escala de las opiniones (valores enteros entre el 0 y el 1), y especificar las columnas del conjunto (identificador del elemento, identificador del usuario y valoración), se usó primeramente KNNWithMeans \cite{scikit_surprise} como algoritmo de filtrado colaborativo básico, que tiene en cuenta las valoraciones promedio de cada usuario.

Tras separar el conjunto en las particiones de entrenamiento (80 \% del conjunto) y prueba (20 \% restante), y entrenar el modelo y predecir con este algoritmo, se usó ALS (mínimos cuadrados alternos, por sus siglas en inglés) para después evaluar la precisión a través del RMSE (error cuadrático medio, por sus siglas en inglés), como se describirá en la fase de evaluación.

A partir de aquí, el sistema funciona de forma análoga al sistema basado en contenido, devolviendo identificadores de municipios recomendados para uno dado.

Todo ello se realizó en el archivo \texttt{users\_recommendations.py}.

Sin embargo, y como se abordará en el apartado de evaluación, este enfoque no dio los resultados deseados, teniendo un rendimiento inferior al esperado –quizás debido a la generación artificial de los usuarios–. Por este motivo, se decidió explorar el segundo tipo de sistemas de recomendación de filtro colaborativo.

\subsubsection{Sistema de recomendación de filtro colaborativo basado en ítems}

En estos sistemas, las recomendaciones se hacen teniendo en cuenta la similitud de los elementos valorados por otros usuarios con respecto a las valoraciones del usuario interesado. Se buscan elementos similares a los gustos comunes de usuarios cercanos. Si bien este método puede funcionar peor para usuarios con gustos muy específicos (usuarios de nichos, segmentos caracterizados por pocos usuarios con gustos comunes), el filtro colaborativo basado en ítems habitualmente ofrece recomendaciones mejores y más rápidas~\cite{filtro_colaborativo_2}. En el caso de este trabajo, se consideró que los municipios españoles reúnen las suficientes características como para poder ser divididos en un número relativamente pequeño de grupos, lo que favorece la ausencia de usuarios de nicho, por lo que se consideró un enfoque apropiado.

Así, para predecir la valoración (uno en caso de que la valoración sea positiva; es decir, el usuario expresa que le gusta el municipio, y cero en caso de que este exprese su rechazo) de un lugar para un determinado usuario, calcularemos la predicción de la valoración de un municipio $i$ por un usuario $u$ de la siguiente manera:

$$pred(u, i) = \frac{\sum_{v \in usuarios \ que \ han \ valorado \ i} sim(u, v) \cdot r(v, i)}{\sum_{v \in usuarios \ que \ han \ valorado \ i} |sim(u, v)|}$$

Donde:

$pred(u, i)$: es la predicción de la puntuación del usuario $u$ para el municipio $i$.

$sim(u, v)$: es la similitud entre los usuarios $u$ y $v$, medida por su similitud coseno.

$r(v, i)$: es la evaluación del municipio $i$ por el usuario $v$, que toma 1 si le ha gustado, y 0 en caso contrario.

Este procedimiento es análogo a la frase \guillemotleft la gente a la que le gusta el elemento X, como el usuario actual, también tiende a valorar positivamente el elemento Y, que el usuario aún no ha valorado, por lo que debería probarlo\guillemotright. El filtrado colaborativo basado en ítems tiene menos errores que el filtrado colaborativo basado en usuarios y, como su modelo menos dinámico se calcula con menos frecuencia, se puede representar en una matriz de menor tamaño, por lo que su rendimiento es superior al de los sistemas de filtrado colaborativo basados en usuarios~\cite{filtro_colaborativo_3}.

Además, para ofrecer la mayor cantidad de resultados posible y la mejor experiencia de usuario, al poder descubrir una elevada cantidad de municipios, se decidió hacer uso de una de las técnicas más usadas en recomendadores por su elevado éxito: Los sistemas de recomendación híbridos. De esta manera, a la lista de municipios sugerida según el cálculo anterior –es decir, según los gustos de otros usuarios similares–, se añadieron municipios similares por contenido a los ya valorados positivamente por el usuario. Como se detallará en la sección de despliegue, esta lista se computa una vez al día para todos los usuarios que alguna vez han valorado algún artículo en el sistema.

\section{Evaluación}

De nuevo, debemos hacer distinción entre los dos sistemas usados:

\subsubsection{Sistema de recomendación basado en contenido}

Se usaron diversas métricas para comprobar el funcionamiento de este sistema:

En primer lugar, la puntuación de Ridge devuelve el coeficiente de determinación de la predicción, siendo mejor cuanto mayor es, y estando su límite superior en 1. Se basa en la creación de un modelo de regresión lineal mediante cuadrados mínimos con norma L2. Su valor no está acotado inferiormente y puede tomar valores negativos, ya que el modelo puede ser arbitrariamente malo. Entre esta fase y su habitual vuelta a la fase de modelado anterior, se probaron las siguientes clases como base del modelo: renta bruta media, que obtuvo una puntuación de 0,62; tasa de actividad, que obtuvo una puntuación de 0,85; tasa de empleo, que obtuvo una puntuación de 0,91; y población, que obtuvo una puntuación de 0,96, siendo por tanto la variable elegida.

Por otra parte, se tomó un subconjunto de 1000 municipios, que se dividieron tomando el 80 \% de ellos en un conjunto de entrenamiento, y un 20 \% en un conjunto de prueba. Tras realizar un etiquetado manual de municipios parecidos, se usó el conjunto de prueba para calcular su RMSE, obteniendo un valor de 0,19, dentro de los objetivos del trabajo.

Además, se realizaron diversas pruebas para comprobar la fiabilidad del sistema de forma cualitativa, y ver si los resultados devueltos eran coherentes y se adecuaban a la expectativa del sistema: Se observó cómo, en todos los casos (100 pruebas formales), el sistema recomendó municipios de características similares; por ejemplo, municipios medianos - grandes como recomendación sugerida para otros municipios del mismo tamaño, municipios pequeños como recomendación sugerida para otros municipios del mismo tamaño, municipios costeros como recomendación sugerida para otros municipios costeros y municipios cercanos a la montaña para otros municipios cercanos a la montaña.

En general, la calidad de las recomendaciones observadas fue muy satisfactoria, tanto para el autor como para los sujetos que evaluaron cualitativamente el sistema una vez desplegado, como se describe posteriormente.

\subsubsection{Sistema de recomendación de filtro colaborativo basado en usuarios}

Por otro lado, el sistema de recomendación de filtro colaborativo basado en usuarios no alcanzó una fiabilidad adecuada en sus predicciones. La estimación del error cuadrático medio tras aplicar el algoritmo KNNWithMeans basado en la similitud coseno era demasiado grande (0,44), por lo que se decidió volver a la fase anterior y usar como método BaselineOnly con ALS y 3 ciclos completos (\textit{epochs}). De esta manera, se obtuvo un RMSE mucho menor, de 0,04, por lo que en principio los resultados eran más prometedores.

Sin embargo, al realizar las predicciones del conjunto de evaluación, los resultados no fueron satisfactorios. Las probabilidades de recomendar o no un municipio para un usuario similar a otro no superaban el 52 \%, quedando demasiado cerca a criterio del autor de una recomendación puramente azarosa.

Se pensó que podría deberse a un tamaño demasiado pequeño en el conjunto de muestra inicial (2000 valoraciones), por lo que se optó por aumentar el número de valoraciones de cada usuario hasta las 1000 por tipo de valoración, resultando en 20 000 valoraciones. Pese a ello, los resultados tras entrenar, probar y evaluar el modelo fueron similares, no alcanzando un 53 \% de precisión en las recomendaciones. Tras aumentar el conjunto de datos de nuevo por diez, creando 10 usuarios nuevos réplica de los anteriores y contar con 200 000 valoraciones, solo se aumentó el tiempo de cálculo necesario, obteniendo los mismos resultados en cuanto a su precisión, muy similar a la de recomendaciones aleatorias.

Se consideró que este hecho puede haber estado relacionado con el tamaño de la muestra (pocos usuarios y pocas valoraciones) o con los criterios arbitrarios elegidos para separar a los usuarios, que no reflejen adecuadamente el comportamiento y valoraciones de usuarios reales.
 
\subsubsection{Sistema de recomendación de filtro colaborativo basado en ítems}

Por su parte, el sistema de filtrado colaborativo basado en artículos obtuvo un RMSE de 0,25 en el conjunto de prueba, generado con 100 valoraciones aleatorias sobre un conjunto de 300 municipios aleatorios, parecidos por pares, siguiendo una proporción 80-20 para cada conjunto de datos (entrenamiento y prueba, respectivamente).

Para asegurar una calidad suficiente, las recomendaciones se tienen en cuenta si el valor de la predicción supera el umbral de 0,7. Además, al igual que en el recomendador de contenido, se realizó la misma evaluación cualitativa formal, consistiendo en 100 pruebas a 10 usuarios distintos (10 pruebas con cada usuario).

Tras alimentar el sistema con sus valoraciones, debían juzgar si las predicciones les resultaban útiles o no. Todos ellos consideraron que las predicciones mostraban municipios similares a aquellos por los que habían expresado su preferencia en el pasado, considerando la prueba satisfactoria.

\section{Despliegue}

Siguiendo con los objetivos de este trabajo, se encuentra una forma de interacción cómoda, centrada en el usuario, que le permita descubrir municipios en base a sus gustos y a los de otros usuarios. Para ello, se considera crear un sitio Web público como la mejor forma de lograrlo. El sitio Web está disponible en \url{www.dondeteesperan.es}, y cuenta con las siguientes características:

En primer lugar, el usuario dispondrá de dos formas para descubrir lugares en base a sus gustos: a través de un cuestionario, y a través de la introducción de una entrada para el sistema de recomendación basado en contenido.

\subsubsection{Cuestionario}

Con la intención de permitir a los usuarios encontrar municipios en base a una búsqueda de criterios que puedan ser de su interés, se diseñan las siguientes 14 cuestiones y filtros, todos ellos opcionales:

\begin{itemize}
    \item Filtrado por comunidad autónoma a la que pertenece el municipio.
    \item Filtrado por provincia a la que pertenece el municipio.
    \item Filtrado por cercanía a la playa del municipio, preguntado en escala de tipo Likert de 5 puntos, donde cada valor se corresponderá con un valor de línea de costa asignado según la heurística utilizada.
    \item Filtrado por cercanía a la montaña del municipio, análogo al anterior.
    \item Filtrado por servicios sanitarios en el municipio, en función de si se desean: centros de salud, centros de salud con servicios de urgencias, hospitales, o todo lo anterior.
    \item Filtrado por servicios educativos en el municipio, en función de si se desean: colegios, universidades o ambos.
    \item Filtrado por cantidad de población del municipio, en función de si el tamaño deseado entra en alguno de los siguientes intervalos: menos de 1000 habitantes, entre 1000 y 5000 habitantes, entre 5000 y 10 000 habitantes, entre 10 000 habitantes y 100 000 habitantes, entre 100 000 habitantes y 250 000 habitantes, y más de 250 000 habitantes.
    \item Filtrado por temperatura media en invierno en el municipio, ofreciendo los siguientes intervalos: menos de 5º C, entre 5 y 10º C, entre 10 y 15º C, y más de 15º C.
    \item Filtrado por temperatura media en verano en el municipio, análogo al anterior, con los siguientes intervalos: menos de 20º C, entre 20 y 22º C, entre 22 y 25º C, y más de 25º C.
    \item Filtrado por altitud del municipio, incluyendo los siguientes intervalos: menos de 250 metros, entre 250 y 500 metros, entre 500 y 750 metros, y más de 750 metros.
    \item Filtrado por distancia a la capital de provincia, con los intervalos: menos de 5 kilómetros, entre 5 y 15 kilómetros, entre 15 y 25 kilómetros, entre 25 y 35 kilómetros, entre 35 y 50 kilómetros, y más de 50 kilómetros.
    \item Filtrado por renta per cápita anual del municipio, dando los intervalos: menos de 20 000 €, entre 20 000 y 25 000 €, entre 25 000 y 30 000 €, y más de 30 000 €.
    \item Filtrado por cobertura de Internet disponible en el municipio, con las opciones: Cobertura móvil 3G en más de la mitad del territorio, cobertura móvil 4G en más de la mitad del territorio, cobertura de fibra óptica de 30 MB en más la mitad del territorio, cobertura de fibra óptica de 100 MB más de la mitad del territorio o cobertura móvil 4G y fibra óptica de 100 MB más de la mitad del territorio
    \item Filtrado por preferencia del usuario entre naturaleza o servicios comerciales, que filtra por los lugares de tipo \guillemotleft sitios\_actividad\guillemotright, \guillemotleft sitios\_ocio\guillemotright\space y \guillemotleft sitios\_natural\guillemotright\space para la primera opción, y por municipios que tengan lugares de tipo \guillemotleft sitios\_comercio\guillemotright, \guillemotleft sitios\_servicio\guillemotright\space y \guillemotleft sitios\_catering\guillemotright\space para la segunda.
\end{itemize}

Además, se consideró necesario añadir una página de \guillemotleft Metodología\guillemotright\space al sitio Web, que explica cómo se ha realizado este trabajo, y que incluye las siguientes notas en relación con el cuestionario anterior –además de asuntos ya descritos en esta memoria, como el cálculo de las heurísticas para las distancias o la toma de valores climáticos promedio en febrero y julio para representar el invierno y el verano, respectivamente–:

\textit{En todos los rangos del cuestionario se incluye el/los extremo/s del intervalo correspondiente, por ejemplo: \guillemotleft Municipios de menos de 10 000 habitantes\guillemotright, incluye a los municipios que tienen 10 000 habitantes, de forma análoga con el ejemplo acotado inferiormente (\guillemotleft Municipios de más de 10 000 habitantes\guillemotright\space)}.

\textit{La denominación \guillemotleft colegios\guillemotright\space incluye indistintamente a \guillemotleft colegios e institutos\guillemotright\space (Centros de educación primaria y secundaria), dado que los datos del Ministerio de Educación no realizan esta distinción.}

\subsubsection{Programación de la aplicación}

Dentro de la creación de la página Web, se destacan como aspectos relevantes los siguientes:

\begin{enumerate}
    \item Al igual que en la fase ETL (extracción, transformación y limpieza), en la aplicación se hace uso de archivos no añadidos al repositorio público, que mantienen las variables de entorno con credenciales y otros secretos como medida de seguridad.
    \item Se especifica la configuración de la instancia de Google App Engine en la que se ejecutará la aplicación.
    \item Se mantiene el modelo y la base de datos con la que trabajará la aplicación, en formatos CSV y JSON por conveniencia según el procesamiento a realizar. Es importante destacar que ambos archivos solo son accesibles por el servidor, y no se exponen al público para evitar una fuga de datos.
    \item La API maneja únicamente el identificador del municipio para realizar los procesamientos necesarios (cálculo de similitud, búsqueda con filtros aplicados, búsqueda de sugerencias con los sistemas de recomendación, obtención de sugerencias aleatorias...), y es en el último momento, cuando se ha determinado qué municipio se presentará, cuando se resuelve la información y se extrae la necesaria de la base de datos. De esta manera se consigue minimizar la latencia y mejorar la experiencia de uso.
    \item Como se explica posteriormente, debido a las limitaciones para operar con archivos de texto de una Plataforma como Servicio como Google App Engine, se ha programado una API REST sencilla que realiza operaciones de lectura y escritura en archivos CSV en un servidor tradicional de pila LAMP. De esta forma, se conserva la posibilidad de tener archivos de fácil manejo por los sistemas de recomendación, a la par que se evita el acoplamiento con los servicios de la plataforma de computación en la nube elegida, y se reduce el coste de operación de la misma.
    \item Se realizan operaciones de homogeneización de los nombres de los municipios, de forma que la página de información de cada uno pueda estar en un directorio estático conocido y permanente.
    \item Como estas páginas son estáticas, se incluye un método para regenerarlas cuando se necesite, así como otro para probar los cambios en un municipio de ejemplo, sin necesidad de regenerar las miles de páginas estáticas de cada municipio. Como son parte de la programación y depuración, estos métodos se han configurado para que solo funcionen en el entorno local.
    \item Las páginas de los municipios, así como cualquier otra página que reciba información dinámica, se han construido mediante plantillas de Jinja2~\cite{jinja2}, que favorecen la reutilización y escalabilidad mediante componentes de código.
    
\end{enumerate}

Una descripción exhaustiva de cada archivo que compone el directorio Web, y una explicación más detallada de los aspectos anteriores pueden encontrarse en el Anexo \ref{programador} \guillemotleft Documentación técnica de programación\guillemotright. 

\subsubsection{Infraestructura}

Muchos de los retos del sistema han venido dados por el uso de una plataforma como servicio para desplegar la aplicación para usuarios finales. Si bien esto fue elegido por aportar muchas ventajas (como la ausencia de la gestión de la infraestructura, o el pago de los recursos únicamente usados y no por adelantado o por tarifas con cuotas mínimas), es cierto que a cambio se pierde control de configuración y de acceso al sistema.

En particular, como una plataforma como servicio abstrae la infraestructura subyacente, no es posible conseguir que los archivos de Python lean y escriban en ficheros de texto, como sí lo hacían en el entorno local. La alternativa es usar otras APIs de Google para la persistencia de datos, algo que se descartó por desviarse del objetivo fundamental de este trabajo, y por perder la adaptabilidad que brinda usar archivos de texto multipropósito (como texto plano, CSV o JSON) como medios de persistencia de bajo tamaño. Además, no se consideró deseable acoplar la persistencia de datos a la infraestructura de Google, ya que una gran ventaja del \textit{Cloud Computing} es la capacidad de migrar de un proveedor a otro con mínimos cambios en la configuración del despliegue.

Así, se relegó la persistencia al servidor de pila LAMP, del que ya disponía el alumno, y del que tiene pleno control para poder escribir código PHP que lea y actualice los ficheros de texto. De esta manera, este segundo servidor ejerce de base de datos distribuida y es la única fuente de verdad, intercambiando las operaciones a realizar y sus resultados mediante la API HTTP creada a tal fin y usada por la aplicación de Python. Este es uno de los principios de las arquitecturas diseñadas como microservicios, que favorecen el desacoplamiento entre sistemas su operación distribuida de forma transparente para el usuario.

Otro aspecto relevante a destacar tiene que ver con los recursos del sistema. Por defecto, las aplicaciones desplegadas en App Engine se alojan en instancias de tipo \guillemotleft F1\guillemotright, que tienen un límite de 256 MB de memoria RAM y 600 MHz de CPU. Pese a que el funcionamiento era totalmente correcto y rápido en el entorno local, a la hora de desplegarlo a producción (el entorno de App Engine), la aplicación se mostraba lenta y errática, no pudiendo completar en torno al 40 \% de las peticiones. Tras inspeccionar los archivos de registro de errores de App Engine, se llegó a la conclusión de que el rendimiento de Python manejando los ficheros de gran tamaño de la base de datos de municipios era el causante de los problemas, ya que agotaban los recursos de la instancia y provocaban que Google la abortara, resultando en un error HTTP 500, error del servidor.

Este aspecto quedó totalmente subsanado cuando, desde el archivo \guillemotleft app.yaml\guillemotright\space se escaló el sistema, especificando \guillemotleft F4\_1G\guillemotright\space como instancia a ser desplegada, que ofrece 2048 MB de memoria RAM y 2,4 GHz de CPU.

Además, las principales plataformas de \textit{Cloud Computing} abaratan los costes para el desarrollador ofreciendo pago por uso, por lo que cuando el servicio no está siendo accedido se destruye su instancia para que no consuma recursos. La contrapartida de esto es el mayor tiempo que toma la primera petición cuando no hay una instancia creada, lo que también puede impactar en la facturación: como cada instancia está activa unos minutos desde que se crea, y se factura por unidad de tiempo con recursos activos, recibir peticiones periódicas es perjudicial para los costos del sistema. A pesar de ello, como cuando la instancia está destruida la petición lleva un tiempo considerable (varios segundos, afectando a la imagen del producto), se ha aprovechado el segundo servidor de pila LAMP para programar unos trabajos periódicos (\textit{cron jobs}) que realicen unas peticiones a la máquina de App Engine cada hora durante parte de la mañana y parte de la tarde de cada día, a fin de que los usuarios que puedan visitar el sistema en esa franja tengan una mejor experiencia de uso. Dado que el tráfico del sistema es muy bajo, el coste adicional de esta medida se ha considerado despreciable para los beneficios que puede conllevar.

\subsubsection{Aspectos de seguridad técnica y jurídica}

Además de garantizar el correcto funcionamiento de la aplicación Web, se han querido también aplicar los conocimientos de las asignaturas \guillemotleft Fundamentos de Ciberseguridad\guillemotright\space y \guillemotleft Derecho en Seguridad de Datos\guillemotright\space para garantizar la seguridad técnica y jurídica del proyecto. Además de las medidas técnicas ya señaladas (establecimiento de los permisos mínimos imprescindibles para realizar las operaciones, contraseñas y credenciales seguras no expuestas en el código, uso de claves renovables o sistemas de copias de seguridad automáticos –activados en los dos servidores y disponible en el código a través del control de versiones del repositorio–), se desean comentar también las medidas tomadas desde un punto de vista de la seguridad jurídica.

Es importante señalar que el sitio Web está configurado para redireccionar siempre a la versión HTTPS, cifrando el tráfico intercambiado como medida de seguridad, y convirtiendo las URL sin \guillemotleft WWW\guillemotright\space a URLs con ellas para facilitar la uniformidad de los recursos y garantizar la seguridad de los usuarios.

Por otra parte, para identificar a los usuarios y sus acciones –estas son, las valoraciones con las que se alimenta el sistema de recomendación colaborativo basado en ítems–, es necesario utilizar una \textit{cookie}. Esto es un pequeño fichero de texto que se almacena en el navegador Web del usuario, y que almacena el identificador único asignado a ese usuario. Sin embargo, dado que el sistema está orientado hacia usuarios europeos, es de aplicación el Reglamento General de Protección de Datos, transpuesto a la legislación española por la Ley Orgánica 3/2018, de 5 de diciembre, de Protección de Datos Personales y Garantía de los Derechos Digitales, que, entre otras cosas, establece la obligación de los prestadores de servicio de recabar el consentimiento informado explícito del usuario antes de realizar operaciones de este tipo~\cite{GDPR_ESP}.

Así, el identificador de usuario –y, por lo tanto, el almacenamiento de la \textit{cookie} en su navegador– solo ocurre cuando este ha dado su consentimiento en una ventana emergente que aparece cuando el sitio carga por primera vez. Esta ventana está gestionada por TermsFeed~\cite{TermsFeed}, y para no volver a mostrarse una vez el usuario ya ha respondido, almacena una \textit{cookie} de tipo técnico con las preferencias manifestadas por el usuario. Solo en el caso de que el usuario otorgue su consentimiento de seguimiento para fines de personalización, se recogerán sus valoraciones y se almacenará la \textit{cookie} propia en su navegador. Además, si el usuario permite el seguimiento para fines analíticos, también se almacenarán \textit{cookies} de Google Analytics~\cite{analytics}, para entender mejor el comportamiento de los usuarios por el sitio Web.

Por su parte, también es necesario disponer de una política de privacidad y \textit{cookies}, que explique transparentemente cómo el sitio Web y su responsable manejan la información de los usuarios. Para ello, se ha contado con el apoyo de la herramienta de Zimrre \& Freelance~\cite{privacidad}.

\subsubsection{Evaluación de la aplicación}

Para evaluar la actividad de despliegue se llevaron a cabo entrevistas con 10 personas de diferente perfil técnico y social, a las que se pidió utilizar el sitio Web sin indicaciones mientras se observa su comportamiento e impresiones. Todos los participantes mostraron su interés en el sistema, y apreciaron su utilidad y facilidad de uso. En particular, el 100 \% de ellos completó las tareas que deseaban realizar cuando conocieron el sistema, y el 85 \% de ellos las completó en menos de 5 minutos cada una.

Además, gracias a su retroalimentación cualitativa y a su observación, se pudieron extraer las siguientes ideas de mejora:

\begin{itemize}
    \item Mejorar el rendimiento del sistema en App Engine, dado que muchas tareas exigían de reintentos por los comentados fallos en el servidor.
    \item Bloquear el desplazamiento lateral en dispositivos táctiles, como tabletas, en el cuestionario para mejorar la navegación.
    \item Reformular preguntas y opciones del cuestionario que presentaron más dudas.
    \item Aclarar el funcionamiento de los controles de desplazamiento para facilitar su uso.
    \item Mejorar la experiencia de usuario en dispositivos móviles.
    \item Modificar las reglas de exclusión de imágenes extraídas automáticamente de Wikipedia para evitar mapas como fondo de página en algunos municipios.
\end{itemize}

Todas ellas fueron incluidas posteriormente en el sistema y se encuentran actualmente publicadas.
