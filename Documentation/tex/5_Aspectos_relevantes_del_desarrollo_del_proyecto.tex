\capitulo{5}{Aspectos relevantes del desarrollo del proyecto}

En este apartado se recogen los hechos más significativos del desarrollo del proyecto.

\section{Ciclo de vida}

Como se ha comentado en el capítulo anterior, la metodología de trabajo elegida ha sido CRISP-DM, que se ha usado como hilo conductor del trabajo. A continuación, se procede a comentar lo más representativo de cada etapa, junto con el trabajo que se ha llevado a cabo en cada una:

\section{Entendimiento de negocio}

Como parte de esta etapa, se definieron los ocho objetivos detallados en la segunda sección de esta memoria, y se definieron los siguientes requisitos para poder alcanzar dichos objetivos:

\begin{enumerate}
    \item El sistema deberá recoger las preferencias del usuario, en el sentido de sus gustos personales, para poder mostrar municipios que las satisfagan.
    \item Estas preferencias se expresarán a través de un formulario de diversas preguntas, que permitan segmentar los municipios candidatos hasta encontrar los que reúnan las características deseadas.
    \item Se implementará un sistema de recomendación basado en contenido, para lo que se deberá recoger la entrada del usuario explícitamente. Adicionalmente, se podrá recoger la entrada implícitamente en otros casos de uso.
    \item Se recogerá también el gusto o no de los usuarios por el resultado ofrecido, de manera que se puedan tener datos para construir un sistema de recomendación basado en filtro colaborativo.
    \item Se explorará un sistema de recomendación basado en filtro colaborativo.
    \item Se recopilarán datos significativos basándose en otros trabajos relacionados y en los que el autor considere relevantes, que permitan crear perfiles detallados para cada municipio, de entre los conjuntos y extracciones de datos disponibles para el ámbito del proyecto. En particular, siempre que sea posible se incluirán datos de: nombre completo del municipio, provincia y comunidad autónoma a las que pertenece, número de habitantes, superficie, densidad de población, disponibilidad de servicios educativos (colegios y universidades), disponibilidad de servicios sanitarios (centros de salud y hospitales), datos climáticos históricos básicos, datos de renta y empleo básicos, datos de venta y alquiler de viviendas en el municipio y en su provincia, cercanía a la capital de provincia, cercanía a la costa y a la montaña, altitud, datos de cobertura de Internet en el municipio o en su provincia, principales servicios de los que dispone, ubicación y extractos de información textual y fotográfica.
    \item El sistema contará con, al menos, un 70 \% de bondad en las métricas elegidas, de forma que se pueda alcanzar una fiabilidad adecuada.
    \item Se diseñará el sistema de forma escalable; es decir, se considerará desde su diseño en posibles repeticiones automáticas del proceso de extracción, transformación y limpieza de datos, posibles adiciones o supresiones de la base de datos de municipios españoles, posibles incorporaciones, modificaciones o supresiones de datos para uno o varios municipios, posibles traducciones o posibles casos de uso similares a los existentes y potencialmente deseables.
    \item Se asignará responsabilidad única a los componentes informáticos que formen el trabajo.
    \item Se presentará un resumen de los datos de cada municipio al usuario.
    \item Se creará un sistema versionado, de fácil despliegue, actualización e interacción.
    \item Se creará un sistema con la privacidad y la seguridad en mente, que no expondrá secretos ni credenciales, que aplicará las mejores prácticas estándar para el ámbito del proyecto, que no recogerá datos personales y que contará con sistemas de contacto para comunicarse con el responsable del sistema.
    \item Se realizará una evaluación de la utilidad, usabilidad y rendimiento del sistema por parte de usuarios potenciales reales, a fin de detectar puntos de mejora y evaluar las decisiones de diseño e implementación adoptadas.
\end{enumerate}

Como criterios de éxito, además del valor de bondad establecido anteriormente, se requerirá una evaluación positiva, de al menos un 70 \% de satisfacción y un 50 \% de expectativa de uso por parte de los usuarios potenciales con los que se probará el sistema.

Para desarrollar el sistema se contará con dos ordenadores a disposición del alumno, el servidor personal mencionado anteriormente –con pila de tecnologías LAMP: Linux, Apache, MySQL y PHP–, y la infraestructura propia de Google –con entorno preparado para ejecutar aplicaciones Web de Python– donde se desplegará el resultado final del proyecto.

El desarrollo del trabajo tiene una duración estimada de alrededor de dos meses, y para completarlo se realizará un análisis de viabilidad previa, recolección y adecuación de los datos necesarios, integración de los datos y composición del modelo, evaluación del modelo e interpretación, presentación y despliegue del resultado final.

\section{Entendimiento de datos}

Una vez completado el análisis de viabilidad con éxito, gracias al descubrimiento de trabajos relativamente relacionados y de los conocimientos del autor en aplicaciones Web y su compatibilidad con otros sistemas basados en Python, se procedió a llevar a cabo la recolección y exploración inicial de datos. Para explicarla, se procede a detallar la función y contenido de cada componente software usado en el proceso de extracción. Nótese que, por fines de compartición de conocimiento y adecuación a los estándares del mundo académico y profesional, los nombres de archivos, variables y otros literales ajenos al modelo se expresan en inglés; mientras que los nombres de las columnas o variables propias de las fuentes de datos mantienen su nombre original en español. Por otra parte, en el caso de nombres escritos en idiomas co-oficiales se ha procurado mantener la convención del Instituto Nacional de Estadística, o la usada mayoritariamente entre las fuentes, en caso de conflicto entre varios nombres alternativos procedentes de diversas fuentes. Es importante recordar que se parte de la base de datos de municipios del INE en formato XLSX, que ha sido convertida a CSV y cuenta con las siguientes columnas: código numérico del municipio, nombre del municipio y provincia a la que pertenece. A continuación, se exponen en orden alfabético los componentes software usados en esta fase. En todos ellos se pasan como parámetros las ubicaciones de los ficheros con los que trabajan:

\begin{enumerate}
    \item \textbf{.env.} Archivo que contiene las credenciales para las APIs de PositionStack, OpenWeather y GeoApify, cuyo uso se detallará en los ficheros correspondientes, y que permite inyectarlas como variables de entorno para que su valor no figure en el repositorio.
    
    \item \textbf{add-coast-mountain-line-to-municipality.py.} Añade a cada municipio la línea en la que se encuentra hasta la costa y la montaña. Con las siguientes particularidades:

    Se ha empleado la heurística detallada en la tercera sección, ``Conceptos teóricos'', de esta memoria, por la que los municipios de las provincias costeras presentan la menor distancia a la costa posible (1), y que esta aumenta conforme aumenta la distancia a las provincias costeras desde las provincias a las que pertenecen los municipios, tomando los valores 2 y 3; 2 para los municipios que se encuentran en provincias limítrofes con provincias costeras; y 3 para los limítrofes con estas provincias. En los casos restantes –es decir; en aquellos en los que la distancia a una provincia costera española es superior a dos provincias, o la dificultad para llegar en línea recta es grande, se toma el valor arbitrario 0 para indicar la máxima lejanía. Es el caso de Ávila, Cáceres, Guadalajara, Salamanca y Toledo. Como excepciones, a La Rioja, Soria y Huesca se les asignó la segunda línea costera a pesar de corresponderles la tercera por su facilidad para llegar por carretera, al tratarse de provincias limítrofes a otras de pequeño tamaño.

    De forma similar, se ha considerado que un municipio está lo más próximo posible a la montaña (valor 1) si en su provincia se encuentran los principales sistemas montañosos españoles: Cordilleras Béticas, Cordillera Cantábrica, Cordillera Costero-Catalana, Islas Canarias, Macizo Galaico, Meseta Central, Pirineos, Sistema Central y Sistema Ibérico. Los municipios en cuyas provincias se encuentran sistemas montañosos significativos pero no contenidos en la lista anterior, o cuyas provincias son limítrofes con las anteriores, se relacionan con el valor 2; mientras que el valor 3 queda reservado para los municipios de provincias colindantes con las anteriores. De nuevo, el valor 0 se usa en provincias especialmente planas de acuerdo con el informe de rugosidad anteriormente mencionado, y otros datos del relieve español \cite{relieve}. En particular, estas provincias son las de Badajoz, Cádiz, Huelva, Sevilla, Valencia, Valladolid y la ciudad autónoma de Ceuta.
    
    \item \textbf{add-income-to-municipality.py.} Este archivo añade los datos de renta per cápita (procedentes de los CSV y XLSX –convertidos a CSV previamente–) de la Agencia Tributaria para todos los municipios excepto los pertenecientes a las comunidades de País Vasco y Comunidad Foral de Navarra, donde los datos se toman de los organismos Eustat y Hacienda Foral, respectivamente.

    Es importante mencionar que, para los municipios de menos de 1.000 habitantes, la Agencia Tributaria facilita datos agregados en vez de individuales, utilizando la denominación ``Agrupación municipios pequeños''. Como se detallará posteriormente, para los municipios que encajan con este tamaño se tomará este valor como su renta per cápita.

    Para el municipio navarro Castillonuevo, Hacienda Foral no facilitó datos por secreto estadístico, al tratarse de un municipio con 18 habitantes según el padrón de 2021. Por este motivo, se tomaron los datos del municipio más cercano: Bigüézal, de 56 habitantes según el mismo padrón.

    \item \textbf{add-province-employment-to-municipality.py.} Añade a cada municipio los datos de empleo (tasa de actividad, tasa de empleo, tasa de desempleo y número de empleos disponibles) de la provincia a la que pertenece cada uno.
    
    \item \textbf{add-province-prices-to-municipality.py.} Añade a cada municipio los datos de viviendas (precio medio del metro cuadrado a la venta, precio medio de las viviendas a la venta, precio medio del metro cuadrado en alquiler y precio medio de las viviendas en alquiler) de las provincias a los que pertenecen.
    
    \item \textbf{add-region-to-municipality.py.} Añade a cada municipio la comunidad autónoma a la que pertenece.
    
    \item \textbf{build-municipality-human-name.py.} Construye para cada municipio su nombre común; es decir, convierte las partes separadas por comas o paréntesis (\textit{Coruña, A}; \textit{Hospitalet de Llobregat, L'}) en la parte inicial del nombre (\textit{A Coruña}; \textit{L' Hospitalet de Llobregat}). Es necesario para poder establecer comparaciones entre distintas fuentes de datos, ya que casi ninguna utiliza el código numérico identificativo del INE.
    
    \item \textbf{clean-schools-municipality.py.} Dado que los datos de centros de educación primaria y secundaria que facilita el Ministerio de Educación corresponden a entidades singulares de población y provincia, y no a municipios, una vez se ha realizado la asociación de cada entidad singular al municipio correspondiente, este archivo limpia el fichero CSV para eliminar las entidades singulares y contar solo con municipio, número de colegios y provincia a la que pertenece.
    
    \item \textbf{complete-nsi-codes.py.} Añade los ceros iniciales a los códigos del INE de menos de cinco cifras, que pueden perderse en algún tratamiento que opera con ellos tratándolos como enteros en vez de como cadenas de texto. El nivel de cinco cifras es el asociado a municipios en la jerarquía del código de 11 cifras del INE.
    
    \item \textbf{complete\_municipalities\_codes.py.} Relacionado con estos códigos, este archivo recibe un fichero CSV y completa los códigos de municipio ausentes con un fichero de referencia que los contenga, basándose en el nombre del municipio y en la provincia a la que pertenece. Ha permitido homogeneizar los datos cuando estos proceden de diversas administraciones o de entidades privadas que no usan los códigos identificativos.
    
    \item \textbf{corsnavigator.php}. Archivo pensado para evadir las restricciones de origen en el control de acceso a sitios Web. El archivo fue subido al servidor personal, y únicamente declara las siguientes cabeceras y carga el contenido de la Web pasada como argumento ``url'' de tipo POST:
    \begin{verbatim}
header('Access-Control-Allow-Origin: *');
header('Access-Control-Allow-Methods: POST, GET, OPTIONS');
    \end{verbatim}
    
    De esta manera, se consigue poder cargar contenido de fuentes externas haciendo una llamada al servidor propio, lo que ha permitido las labores de extracción Web en varias ocasiones.

    \item \textbf{discard-fields.py.} Descarta las columnas especificadas de un archivo CSV, dado que en algunas ocasiones ha sido necesario prescindir de algunos datos de los conjuntos manejados (número de declarantes o posicionamiento del municipio a nivel autonómico en cuanto a su renta bruta media, por ejemplo).
       
    \item \textbf{encode-places.py.} Codifica los sitios más repetidos en un municipio (separados por punto y coma, puede haber hasta tres, según el proveedor de API geográfica usado) como valores binarios en tantas columnas como posibles categorías hay; es decir, en notación \textit{one hot encoding}.
    
    \item \textbf{extract-universities.html}. Extrae el número de centros de educación superior presentes en cada municipio del portal del Ministerio de Educación. A diferencia del resto de extracciones Web, se realizó mediante código JavaScript en un archivo HTML –por tanto, ejecutándose desde el navegador en vez de desde la línea de comandos– por ser el primer caso de Web \textit{scraping} que se llevó a cabo y disponer de mayor familiaridad con las técnicas frente a Python. Sin embargo, esta tecnología fue rápidamente descartada para el resto de casos debido a la mayor comodidad, facilidad de depuración y rendimiento que ofrece la segunda tecnología, que puede ser lanzada desde la línea de comandos y no bloquea el navegador Web.

    El proceso para extraer el número de municipios fue el siguiente: En primer lugar, se debe acceder al Registro de Universidades, Centros y Títulos (RUCT) del Ministerio de Educación, en particular, a la sección que contiene todos los centros, sabiendo que están paginados en 198 páginas.

    \imagen{unis2}{Listado de centros universitarios del RUCT.}

    Posteriormente, y conociendo la URL de cada página del anterior listado, se extrae el código de universidad y el código de centro de cada fila, para finalmente componer la URL de cada centro, que usa ambos códigos. Desde esta dirección es posible saber en qué municipio está cada centro y a qué universidad pertenece, de forma que posteriormente es trivial agruparlos por universidades y municipios, y realizar su sumatorio para devolver el número de universidades (centros de educación superior) por municipio.

    \imagen{unis4}{Página de detalle de cada centro, compuesta a partir de los códigos de universidad y centro anteriores.}
    
    \item \textbf{fix-incomes.py.} Inicialmente se cometió un error en la asignación de rentas de municipios pertenecientes al País Vasco, dado que se confundió renta per cápita con renta familiar, resultando en valores anómalamente superiores que fueron detectados en la parte de exploración de esta fase. Como consecuencia, fue necesario recopilar los datos correctos de renta per cápita de la oficina de estadística vasca, y crear este guión para sustituirlos en el fichero correspondiente.

    \item \textbf{geocode-municipalities.py.} Emplea la API de PositionStack para devolver la latitud y longitud dados un nombre de un municipio y su provincia. Se ha empleado este servicio dado que, tras realizar el análisis de costes de varias alternativas, ofrecía una cuota gratuita considerablemente más generosa que otras alternativas comerciales más conocidas.
    
    Cabe destacar que fueron necesarias dos iteraciones, dado que en la primera se omitió añadir ``, España'' como parte final de la dirección física a geolocalizar, lo que provocó que se devolvieran resultados de otras partes del mundo, como América Latina, Estados Unidos u otros territorios.

    \imagen{darrinward}{Exploración de los datos de geolocalización tras la primera iteración.}

    De nuevo, estos valores incorrectos fueron detectados en la exploración de los datos gracias a la herramienta citada anteriormente, lo que permitió su corrección en la segunda pasada.
    
    \item \textbf{get-distance-province-capital.py.} Este fichero calcula la distancia por carretera desde un municipio (sus coordenadas de latitud y longitud) hasta su capital de provincia (también geolocalizada). El cálculo se realiza por medio de OpenTripPlanner, y en los casos en los que no es posible obtener un resultado (por ejemplo, territorios insulares), se toma la distancia geodésica a través de GeoPy.

    Para realizar estos cálculos se llevó a cabo el siguiente proceso: En primer lugar, es necesario tener geolocalizados todos los municipios, y conocer la provincia a la que pertenecen y su capital. Posteriormente, se descargó la última versión de OTP, distribuida como un fichero JAR (\textit{Java Archive}, por sus siglas en inglés), junto con el mapa de España suministrado por Geofabrik \cite{spain_map} en formato OSM PBF (\textit{OpenStretMap Protocolbuffer Binary Format}, formato binario de ProtocolBuffer –formato de compresión muy eficiente, como se vio en la asignatura ``Infraestructura para el Big Data''– de OpenStreetMap).

    Acto seguido se utilizó la herramienta BoundingBox, de Klokan Technologies \cite{klokan}, para obtener las coordenadas geográficas de la caja que rodea al mapa de España, para convertir el mapa a formato PBF con el siguiente comando de la utilidad de Unix osmconvert, indicando que queremos todas las vías que puedan quedar cortadas por los bordes de la caja:

    \begin{verbatim}
osmconvert spain-latest.osm.pbf -b=-18.39,27.43,4.59,43.99
--complete-ways -o=spain.pbf
    \end{verbatim}

    A continuación, iniciamos el servicio de OTP, que levanta un servidor local de Java en el puerto 8080 de nuestro ordenador. Es importante señalar que estas labores se realizaron en otro ordenador a disposición del alumno, dado que se requería más potencia que la de su máquina personal. En particular, fueron necesarios 28 GigaBytes de memoria RAM para poder arrancar el servidor Java usando el mapa de España completo, debido al tamaño de este. Esto se consiguió ejecutando el siguiente comando desde el directorio que contiene el fichero JAR y el mapa en formato PBF, que asigna dicha memoria a la máquina virtual de Java:

    \begin{verbatim}
java -Xmx28G -jar otp.jar
    \end{verbatim}

    \imagen{otp1}{Inicio del servidor de OpenTripPlanner en la segunda máquina del alumno.}

    Pasados unos minutos, la inicialización del servidor se completa, lo que permite ejecutar el servicio Web de enrutado entre ubicaciones de OTP, que cuenta con la siguiente interfaz gráfica:

    \imagen{otp_ruta}{Interfaz gráfica del servicio de OpenTripPlanner.}

    Como las dos máquinas del alumno se encuentran en la misma red, es posible acceder a este servicio desde la máquina que contiene el repositorio vía su dirección IP. Tras realizar la depuración correspondiente a través de la herramienta de manejo de APIs Postman, se procedió a realizar las solicitudes correspondientes y extraer la distancia del itinerario en kilómetros.

    En el caso de los municipios cuya distancia por carretera no fue posible calcular, fue trivial obtener el cálculo de la distancia geodésica que los separa a través de la biblioteca GeoPy.    
    
    \item \textbf{get-elevation.py.} Dado un fichero con coordenadas geográficas, recupera la altitud sobre el nivel del mar de ese punto a través de la API de OpenElevation.
    
    \item \textbf{get-failed-properties.py.} En el proceso de extracción de datos sobre propiedades inmobiliarias (venta y alquiler de viviendas en municipios y provincias) fallaron municipios de nombre compuesto, por ejemplo ``Alicante/Alacant'', por lo que se usó este archivo para realizar subconjuntos de estos nombres, con la intención de repetir el proceso con el primer nombre (``Alicante''), o con el segundo (``Alacant''), para obtener el mayor número posible de datos. Esto es así ya que no existe consenso sobre qué nombre debe escribirse primero, y depende del criterio –muchas veces arbitrario– de la fuente de datos elegida.
    
    \item \textbf{get-first-name.py.} Relacionado con el propósito anterior, en algunos momentos ha sido necesario extraer el primer nombre en los municipios con varios nombres.
    
    \item \textbf{get-missing-fields.py.} Añade campos de un subconjunto de datos a otros. Se ha usado para unir subconjuntos de datos intermedios sin claves comunes.
    
    \item \textbf{get-municipality-from-singular-entity.py.} Devuelve el municipio asociado a una entidad singular de población, necesario para algunos conjuntos de datos, como los de centros de educación primaria y secundaria del Ministerio de Educación.
    
    \item \textbf{get-places.py.} Devuelve los tres tipos de lugares de interés más frecuentes y el primero de ellos en otra columna, por razones de comodidad y para facilitar su posterior exploración de datos, a través de la API de GeoApify.

    Para ello, se solicita al proveedor los 20 (valor por defecto e involucrado en la facturación del servicio) lugares de interés en un radio de 5 kilómetros alrededor de las coordenadas especificadas de entre las siguientes categorías: comercios, turismo, alojamientos, ocio, naturaleza, servicios, actividades, entretenimiento, hostelería y deporte.

    Cabe destacar que, pese a que se solicitaron las mencionadas categorías, el servicio devolvió también lugares pertenecientes a las siguientes clases que, al igual que en las anteriores, se han traducido al castellano para facilitar su lectura en este trabajo:

    \begin{enumerate}
        \item Edificio. Comprende edificaciones de cualquier clase, habitualmente con fines prácticos.
        \item Acceso limitado. Abarca lugares de acceso al público limitado.
        \item Artificial. Está formada por construcciones humanas.
        \item Acceso. Son lugares de acceso público.
        \item Sin acceso. Son lugares de acceso privado, sin acceso público.
        \item Patrimonio. Abarca lugares de interés patrimonial.
        \item Carretera. Comprende vías de comunicación por carretera.
        \item Cuota. Está formada por lugares que exigen el pago de una cuota de socio para poder entrar.
        \item Comodidad. Son utilidades públicas de conveniencia, como baños públicos.
    \end{enumerate}
    
    Pese a que algunas pueden parecer algo subjetivas y de definición algo abstracta según la documentación del servicio, se consideró beneficioso tenerlas en cuenta para aumentar los datos sobre los municipios.

    El servicio devuelve los, como máximo, 20 lugares de mayor interés encontrados, con los datos de sus categorías correspondientes. A continuación, se crea para cada municipio un diccionario de clave-valor, donde la clave es el tipo de lugar, y el valor es el número de lugares encontrados que pertenecen a dicha categoría.

    Finalmente, estos diccionarios se agrupan y ordenan por valor de forma descendente para devolver los tipos más recurrentes.

    \item \textbf{get-properties-for-rent.py.} Extrae el precio medio del metro cuadrado en las viviendas en alquiler, el precio medio del alquiler y el número de viviendas en alquiler para los municipios consultados del portal inmobiliario Web Idealista.

    Para ello, construye la URL según el convenio usado por Idealista (nombre del municipio y provincia), y realiza una petición idéntica a la que realizaría un navegador Web humano. Esto es debido a las medidas de seguridad de Idealista para evitar tráfico automático, que exigen enviar datos como una \textit{cookie} de sesión o un identificador único de usuario. Estos datos se tomaron realizando una primera petición desde el navegador de forma manual e inspeccionando el contenido intercambiado en las cabeceras de los paquetes de red.

    Posteriormente, extrae los datos mencionados leyendo el contenido de la página y utilizando los selectores HTML adecuados a través de beautifulsoup4. Los valores promedio por superficie se calculan diviendo el precio entre el número de metros cuadrados de cada vivienda, para después realizar su media aritmética; mientras que el precio medio del alquiler se calcula sumando los valores encontrados en la página inicial y diviéndolos entre el número de viviendas encontradas en dicha página. Por otra parte, el número de viviendas viene dado como literal y su extracción es trivial.

    Cabe señalar que, con la intención de evitar causar molestias, disrupciones de servicio o violar los términos y condiciones de este tercero, entre petición y petición el programa realiza una pausa de 2 segundos para asemejarse lo máximo posible a un visitante humano cuyo tráfico pueda manejar el servidor fácilmente.

    Sin embargo, cuando se habían realizado algo más de 4.000 peticiones (el 50 \% de todas las necesarias, dado que se necesita una por municipio) el servicio comenzó a bloquear el tráfico desde la IP del alumno, devolviendo un error HTTP 429, indicando la realización de demasiadas peticiones.

    Por este motivo fue necesario repetir la búsqueda con el subconjunto fallido, dejando pasar un tiempo prudencial tras el que se eliminó la restricción de llamadas.

    \imagen{idealista}{Error HTTP 429 del portal Idealista a partir del 50 \% del proceso durante la primera iteración.}

    Se recalca que esta no es la manera favorita de realizar extracciones de datos para el autor, pero lamentablemente el portal no dispone de una API pública, y fueron varias las opiniones negativas encontradas sobre la elevada dificultad e incertidumbre en el uso de la API privada (previa aprobación del portal), dado que tampoco se encuentra documentada. En cualquier caso, se enfatiza el uso de estos datos para fines exclusivamente educativos, como es la realización de este trabajo académico, y se recomienda la opción de contactar con este tipo de portales de cara a buscar la solución que mejor se adapte a las necesidades del interesado, como comprar los informes de datos del mercado inmobiliario que ellos mismos venden si se desea explotar este aspecto comercialmente.

    La extracción de estos datos se realizó gracias a este programa de forma automática en 10 horas no consecutivas (considerando las dos pasadas necesarias).

    \item \textbf{get-properties-for-rent-fotocasa.py.} De forma análoga al programa anterior, este fichero realiza la extracción de datos del alquiler para municipios usando el servicio de Fotocasa.

    Se usó de forma auxiliar cuando no fue posible recuperar algún dato con Idealista y a modo de prueba comparativa, dado que el coste de crearlo fue considerado residual frente a los beneficios de probar otro proveedor. La formación de la URL del municipio y la extracción de sus datos es prácticamente idéntica, y este portal –aunque con menores datos que el anterior– se comportó mejor a nivel de rendimiento, no limitando las peticiones del alumno en ningún momento (que también estuvieron pausadas, esta vez con un segundo entre ellas, por los mismos motivos anteriores).

    \item \textbf{get-properties-for-sale.py.} Este archivo contiene el mecanismo equivalente al descrito anteriormente para extraer los datos de propiedades inmobiliarias en venta del portal Idealista, tomando los precios medios por metro cuadrado, el precio promedio de las viviendas a la venta y el número total de ellas disponibles por municipio.

    \item \textbf{get-properties-for-sale-fotocasa.py.} Este programa realiza la misma extracción anterior, pero usando el portal Web inmobiliario Fotocasa.

    \item \textbf{get-province-sales-rent-prices.py.} Extrae los precios medios y los precios medios por metro cuadrado de las viviendas en venta y en alquiler de cada provincia a través de las páginas provinciales de Fotocasa. Sigue la misma lógica descrita en los casos anteriores, pero con mucha menos complejidad, ya que únicamente existen 52 provincias frente a los 8.131 municipios españoles.

    \item \textbf{get-weather.py.} Extrae los datos climáticos considerados como más representativos dado un municipio geolocalizado, a través del servicio de OpenWeather.

    Tras considerar varias alternativas, se optó por este servicio por el recuerdo de buenas experiencias pasadas con la cantidad y calidad de los datos, y con la facturación. El análisis de viabilidad previo al comienzo de esta extracción confirmó que era posible obtener los datos deseados gratuitamente con la licencia de estudiantes de esta plataforma. Tras solicitarla por seis meses, se exploró la API, que devuelve los datos climáticos históricos de los que dispone OpenWeather (referente en  observación y predicción meteorológica) para una latitud y una longitud concreta en un día, mes o año concretos.
    
    Las opciones de día (el tiempo histórico para un día dado) o año (el tiempo durante un año concreto) no parecen muy significativas para este proyecto, a diferencia de la opción mensual, que devuelve el tiempo histórico registrado en un lugar durante dicho mes.
    
    Sin embargo, el plan mediano de datos históricos incluido en la licencia de estudiante de este servicio, limita el número de llamadas diarias a 50.000, lo que imposibilita recoger datos de todos los meses para cada municipio y procesarlos después –dado que serían necesarias 12 llamadas por cada municipio, unas 97.572–. Por este motivo y debido a las restricciones de recursos de este trabajo, se optó por recoger datos de dos de los meses más significativos del año en nuestro país: febrero y julio, con el objetivo de representar los datos de invierno y verano de forma acusada.
    
    Usando su API, se recogieron los siguientes datos:
    
    \begin{itemize}
        \item Promedio de las temperaturas mínimas en febrero.
        \item Promedio de las temperaturas máximas en febrero.
        \item Promedio de las temperaturas en febrero.
        \item Promedio de humedad en febrero.
        \item Promedio de la velocidad media del viento en febrero.
        \item Promedio de las precipitaciones mínimas en febrero.
        \item Promedio de las precipitaciones máximas en febrero.
        \item Promedio de las precipitaciones en febrero.
        \item Promedio de porcentaje de nubosidad en febrero.
        \item Promedio de número de horas de sol en febrero.
    
        \item Promedio de las temperaturas mínimas en julio.
        \item Promedio de las temperaturas máximas en julio.
        \item Promedio de las temperaturas en julio.
        \item Promedio de humedad en julio.
        \item Promedio de la velocidad media del viento en juliio.
        \item Promedio de las precipitaciones mínimas en julio.
        \item Promedio de las precipitaciones máximas en julio.
        \item Promedio de las precipitaciones en julio.
        \item Promedio de porcentaje de nubosidad en julio.
        \item Promedio de número de horas de sol en julio.
    \end{itemize}
    
    
    Cabe señalar que el servicio devuelve los datos de temperaturas en Kelvin y los datos de velocidad del viento en metros por segundo. Ambos se convierten a las unidades habituales en España (grados centígrados y kilómetros por hora, respectivamente) para poder realizar adecuadamente las tareas de exploración de los datos.
    
    \item \textbf{get-wikipedia.py.} Recupera para cada municipio un extracto de texto e imágenes representativas de su entrada en Wikipedia. Esta información no se usa en el modelo pero sí en la presentación e interpretación de los resultados para el usuario.

    Para ello, toma el nombre ``humano'' del municipio (sin comas ni paréntesis, como se ha comentado anteriormente) y lo busca junto con el nombre de su provincia en la Wikipedia en español, tomando el primer resultado.
    
    A partir de ahí, extrae el resumen del artículo y las fotografías que aparecen en él, con algunas particularidades: Con el objetivo de mostrar imágenes útiles y visualmente atractivas, descarta aquellas que contienen los siguientes términos en su nombre, sin distinción de mayúsculas o minúsculas: \textit{flag, coat, bandera, escudo, icon, logo, svg}, omitiendo muchas imágenes de escudos, banderas o mapas. Posteriormente, y por razones de rendimiento, toma las primeras 10 imágenes.
    
    En algunos casos es necesario tomar el nombre de la provincia en castellano (``Islas Baleares'', en vez del uso muy extendido ``Illes Balears''), ya que los segundos provocaron mayor número de resultados incorrectos (dado que la enciclopedia devolvía entradas no relacionadas con los municipios, sino con personas, instituciones o empresas).
    
    Dado que, debido al volumen de datos, es un contenido no supervisado y ampliamente cualitativo, su exploración automática ha sido más complicada. Por este motivo, en la sección de ``Metodología'' del resultado final –en la que también se explica en líneas generales cómo se ha realizado el proyecto– se indica este hecho, así como la posibilidad de contactar con el autor para informar de información errónea que no haya podido ser detectada antes.
    
    Los resultados se exportan a un fichero en formato JSON.

    \item \textbf{group-clinics-by-municipality.py.} Toma los datos de centros sanitarios del Ministerio de Sanidad y los agrupa por municipio y provincia, ya que estos vienen desglosados por localidad, haciendo referencia a la entidad singular de población. Posteriormente, descarta las columnas no relevantes, como dirección, código postal o área de salud, para realizar el sumatorio del número de centros de salud por municipio.

    \item \textbf{group-connectivity-by-municipality.py.} Toma los datos de conectividad del Ministerio de Asuntos Económicos y Transformación Digital y los agrupa por municipio y provincia, ya que estos vienen desglosados por entidades singular de población.

    \imagen{cobertura}{Archivo Excel con los datos de cobertura de Internet de banda ancha por entidades singulares de población del Ministerio de Asuntos Económicos y Transición Digital.}
    
    Previamente fue necesario asociar a cada entidad singular de población su municipio, según se ha explicado anteriormente.

    \imagen{entidades2}{Archivo Excel con la relación entre las entidades singulares de población y el municipio al que pertenecen de Francisco Ruiz.}
    
    A continuación, realiza el promedio de los valores porcentuales del territorio de cada municipio cubierto por una determinada cobertura, para las siguientes conectividades: fibra óptica de 30 megabytes de velocidad o superiores, fibra óptica de 100 megabytes de velocidad o superiores, red móvil de tercera generación (3G HSPA) y red móvil de cuarta generación (4G LTE).

    \item \textbf{group-emergencies-by-municipality.py.} Toma los datos del Ministerio de Sanidad y los agrupa por municipio y provincia, ya que estos vienen desglosados por localidad, haciendo referencia a la entidad singular de población. Posteriormente, realiza el sumatorio del número de centros de salud con servicio de urgencias extrahospitalarias por municipio.

    \item \textbf{group-hospitals-by-municipality.py.} Toma los datos de centros sanitarios del Ministerio de Sanidad y los agrupa por municipio y provincia, ya que estos vienen desglosados por localidad, haciendo referencia a la entidad singular de población. Posteriormente, descarta las columnas no relevantes, como dirección, código postal o área de salud, para realizar el sumatorio del número de hospitales de gestión pública o privada por municipio.
    
    \item \textbf{group-schools-by-municipality.py.} Toma los datos de centros educativos del Ministerio de Educación y los agrupa por municipio para realizar el sumatorio del número de centros educativos por municipio.
    
    \imagen{entidades1}{Archivo Excel con los datos de centros educativos por entidades singulares de población del Ministerio de Educación.}
    
    \item \textbf{merge-properties.py.} Fusiona los datos de las distintas iteraciones necesarias para recuperar los datos de propiedades inmobiliarias de los portales Idealista y Fotocasa.
    
    \item \textbf{merge.py.} Fusiona los ficheros con los resultados parciales obtenidos de las distintas extracciones anteriores para formar un fichero que los agrupe a todos y que constituirá el modelo posterior.
    
    \item \textbf{random-sample-csv.py.} Extrae una muestra aleatoria de tamaño variable de un fichero de municipios pasado como argumento, de forma que pueda usarse para probar y depurar las distintas funciones de extracción.
\end{enumerate}

Los archivos descritos se pueden encontrar en el directorio ``Scripts/ETL'' del repositorio entregado, y además de los mismos, también se quiere hacer mención a los siguientes ficheros auxiliares usados durante este proceso y que se subieron al servidor personal del alumno para poder realizar labores de depuración sin incurrir en costes adicionales en los servicios de terceros:

\begin{itemize}
    \item \textbf{geocoder-mock.php.} Contiene un ejemplo de respuesta para un municipio del servicio usado como proveedor de geocodificación, de forma que se pudiera probar la lógica que extrae la latitud y longitud del municipio sin llamar al servicio real.
    \item \textbf{places-mock.php.} Muestra un ejemplo de respuesta para los lugares de interés encontrados en un municipio por el proveedor de servicios de localización, para probar y depurar la lógica relacionada sin alcanzar el servidor de GeoApify, evitando gastos innecesarios.
    \item \textbf{planner-mock.php.} Constituye un ejemplo de respuesta del servicio OpenTripPlanner para el cálculo de ruta entre dos coordenadas geográficas, de forma que se pudiera probar la lógica que calcula la distancia entre un municipio y su capital de provincia sin usar el servicio real.
    \item \textbf{villares.html.} Página de un municipio del resultado software final, utilizada para probar desde un navegador real cómo se visualizaría la página desde Internet –y no desde \textit{localhost}– antes de iniciar la fase de despliegue.
\end{itemize}

Todos los datos han sido recogidos en diciembre de 2022, y hacen referencia a los datos más recientes disponibles a esa fecha, siendo las fechas más distintas a la fecha de recogida las de los datos:

\begin{itemize}
    \item Rentas brutas medias por municipio, pertenecientes al año 2019.
    \item Cobertura de banda ancha por municipio, perteneciente al año 2020.
\end{itemize}
