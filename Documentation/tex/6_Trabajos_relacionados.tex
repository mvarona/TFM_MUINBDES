\capitulo{6}{Trabajos relacionados}

A continuación, se citarán diversos trabajos relacionados analizados durante la elaboración de este trabajo y cuyas características se han tenido en cuenta a lo largo del proyecto.

\section{Repuéblame}

Creado por Javier y Guido García, Repuéblame \cite{repueblame} es un portal en línea que ofrece las características de los 2.248 municipios de Castilla y León. Entre sus puntos fuertes destacan una interfaz limpia y cuidada y varios selectores para filtrar la información, que parece haber sido recogida de los Datos Abiertos de la Junta de Castilla y León \cite{datos_abiertos_jcyl}.

\imagen{repueblame2}{Página principal de Repuéblame}

Destaca también la página de cada municipio, que muestra un mapa y una imagen del lugar, acompañados por un texto generado para cada uno en función de sus características. El servicio otorga a cada localidad una puntuación dentro de un sistema arbitrario no público. Este sistema facilita la lectura y la interpretación de los resultados, pero sesga también la visión del usuario, obligándole a confiar en unos criterios opacos que no tiene por qué compartir.

\imagen{repueblame3}{Página de municipio de Repuéblame}

El pie de página de cada municipio ofrece otros municipios cercanos y otros similares, utilizando quizás un sistema de recomendación basado en contenido como el descrito en este trabajo. Ofrece también una sección para destacar diferentes municipios de selección editorial con periodicidad mensual.

\imagen{repueblame4}{Pie de página de municipio de Repuéblame}

Se cree que el sistema podría carecer de \textit{backend}, realizando todo el procesamiento necesario para el filtrado en el cliente Web, lo que ralentiza la navegación y crea problemas de rendimiento en listados grandes. Además, apoya esta hipótesis la existencia de un archivo JavaScript con la base de datos que maneja la página, y que deja entrever las fuentes que se han podido usar y los diferentes criterios seguidos en cada categoría. Este enfoque se considera peligroso ya que permite extraer muy fácilmente el modelo, pieza clave del sistema, haciendo su copia trivial para un tercero malintencionado.

Por último, se destaca que la página de cada municipio está creada estáticamente, lo que mejora el posicionamiento y descubrimiento del sitio, ya que permite encontrar la información de cada municipio desde cualquier buscador.

El trabajo recibió el primer premio de la categoría Productos y Servicios de la V Edición del Concurso de Datos Abiertos de la Junta de Castilla y León \cite{cyl}.

\section{City Recommender System}

Descrito por Elias Melul en una serie de tres artículos (\cite{US1}, \cite{US2}, \cite{US3}), se trata de un sistema de recomendación basado en contenido para 303 ciudades estadounidenses. Debido a la concreción de los artículos y a la elaboración del proceso, se ha seguido su propuesta para el desarrollo del sistema presentado, que, a pesar de la gran diferencia en tipo y cantidad de datos con los que trabajan (8.131 municipios de muy variados tamaños frente a 303 ciudades), conserva su esencia.

Fruto de la naturaleza de los dos escenarios para los que están pensados ambos sistemas, surgen las principales diferencias. Dado que el recomendador de Elias está pensado para ciudades estadounidenses, la existencia de conjuntos de datos de buena calidad y gran cantidad no ha sido problemático. Destacan indicadores socio-económicos, como la seguridad ciudadana, la calidad de vida o el coeficiente de Gini para medir la desigualdad de las localidades. Son todos ellos datos imposibles de conseguir para los municipios españoles, más allá de las ciudades de mayor tamaño, dentro del ámbito de este trabajo; por lo que ha sido necesario considerar cómo se podría sustituir esa información (difícil de entender en su contexto y comparar intuitivamente), por métricas que sí pueden mostrarse junto a cada municipio y que aportan un valor similar.

En el trabajo presentado, los municipios de menor tamaño representan el reto más importante, ya que suponen una mayor dificultad a la hora de encontrar información de múltiples tipos (entradas en Wikipedia, datos tributarios agregados por municipios de similar tamaño, escasas viviendas disponibles...). Este problema, ausente en el trabajo de E. Melul por el diferente dominio, se ha resuelto contando con los datos de las provincias para aquella información que no se ha podido encontrar en un ámbito más pequeño; por ejemplo, a través del número de empleos disponibles en la provincia o el precio medio del alquiler. También se ha encontrado el caso de datos agregados para municipios de muy poco tamaño, como los datos tributarios, donde se ha usado esta información como valor más fidedigno para representar la realidad de la variable de interés.

Se trata de un sistema que carece de interfaz gráfica, ejecutándose únicamente por línea de comandos; por lo que se debió investigar la viabilidad de conectarlo a una página Web antes de comenzar el desarrollo del proyecto presentado. Se encontró el \textit{micro-framework} Flask de Python como mejor opción para realizar esta tarea.

\section{Apadrina un pueblo}

\imagen{apadrina1}{Página principal de Vanwoow}

Se trata de una iniciativa privada de la empresa de auto-caravanas y otros vehículos de alquiler Vanwoow \cite{apadrina_un_pueblo}, que ofrece un mapa para localizar pueblos adheridos a la iniciativa, y la posibilidad de sumarse como empresa apadrinadora. Esto último hace que marcas importantes aparezcan como patrocinadoras y colaboradoras del proyecto, del que se presupone una menor recolección automática de datos que en los anteriores debido a la aparente curación de los textos y el contenido mostrado.

\imagen{apadrina2}{Mapa de Vanwoow}

En el mapa interactivo se puede filtrar por áreas de pernocta, pueblos, experiencias y rutas; y la página de cada municipio, estática de nuevo para ser indexada por los buscadores, ofrece un listado con los principales servicios que el usuario puede encontrarse allí, centrándose en el escenario turístico de pernocta en áreas habilitadas para ello.

\imagen{apadrina3}{Página de municipio en Vanwoow}

Además, ofrece actividades a realizar en cada municipio y su precio, negocios locales y una sección de comentarios que incentiva la interacción social de la comunidad.

\imagen{apadrina4}{Pasos para anunciar un pueblo en Vanwoow}

Ha recibido varios premios por el carácter innovador de su idea, aunque debido a ser una empresa privada cuyo modelo de negocio gira en torno a la propuesta de descubrir ``micropueblos'' en ``autocaravana, caravana o camper'', se aprecia la curación de contenidos en artículos de blog, en la elaboración de rutas o en la colaboración con los pueblos anunciados, cuyo servicio de promoción se anuncia como una opción para revitalizarlos. Es por ello que se cree que el uso de técnicas de extracción automática de información es limitado y menor que en el caso de otros trabajos estudiados.

\imagen{apadrina5}{Propuesta de valor para municipios de Vanwoow}

\section{Recomendadores de filtro colaborativo}

En cuanto a los recomendadores de filtro colaborativo, se han estudiado los trabajos de Abhinav Ajitsaria \cite{filtro_colaborativo_1} y Jeffery Chiang \cite{filtro_colaborativo_2}, que presentan enfoques complementarios a una misma tarea.

El primero apuesta por un filtrado colaborativo basado en usuarios, mientras que el segundo hace uso de un filtrado colaborativo basado en ítems. Ambos trabajos han sido seguidos para la exploración de este tipo de sistemas de recomendación, aunque destaca el mejor rendimiento del segundo sobre el primero, acorde con lo comentado anteriormente: Los sistemas de recomendación basados en ítems son de menor ayuda para usuarios con gustos muy particulares, pero computan menos operaciones y su velocidad de cálculo es mayor.