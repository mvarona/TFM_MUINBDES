\capitulo{2}{Objetivos del proyecto}

A continuación, se exponen los principales objetivos de este trabajo:

\begin{enumerate}
    \item \textbf{Construir un sistema de recomendación de municipios basado en las preferencias del usuario.} Como parte de este objetivo, se acotará la definición de \guillemotleft preferencias del usuario\guillemotright\space y la expectativa de cómo se han de manifestar y tratar. Además, queda establecido por el objetivo que habrá algún tipo de interacción que permitirá introducir datos y mostrar el resultado correspondiente.
    \item \textbf{Recopilar los datos suficientes como para crear un perfil adecuado de cada municipio.} Este objetivo abarca todas las fases de extracción, transformación y limpieza de datos, con el objetivo de garantizar una adecuada cantidad y calidad en los datos con los que se alimentará al sistema.
    \item \textbf{Crear un modelo que se comporte adecuadamente.} Para conseguir esta meta, el sistema deberá ser capaz de recomendar municipios similares y no necesariamente evidentes, cuya sugerencia realmente aporte una utilidad real, sea pertinente y coherente; es decir, se evaluará el rendimiento del sistema y se ajustará de manera que arroje las mejores métricas posibles.
    \item \textbf{Explorar los dos tipos de sistemas de recomendación fundamentales.} Estos son los basados en contenido y los basados en filtros colaborativos. Se procurará explorar ambas ideas con la intención de buscar los casos de uso que mejor se adaptan a cada una, viendo cuál es la que mejores resultados ofrece y analizando los retos encontrados y superados en ambas.
    \item \textbf{Ofrecer un sistema que pueda ser puesto en producción fácilmente.} El sentido de este objetivo es doble: Por una parte, garantizar la búsqueda de soluciones viables y eficientes, que permitan devolver el resultado de una recomendación fácilmente, sin una demora significativa que impida su uso de forma cómoda. Por otra parte, busca aplicar los conocimientos de las asignaturas de la rama de ciencia de datos, dado que será necesario almacenar grandes cantidades de información y operar con ellas en alguna de las soluciones de infraestructura escalable estudiadas.
    \item \textbf{Crear el sistema velando por las mejores prácticas que permitan su escalado y mejora continua.} Con este fin se pretende que todas las fases del proyecto –diseño, extracción, transformación, limpieza, modelado, programación, implementación y despliegue– apliquen las mejores prácticas posibles, cada una en su ámbito, para garantizar el escalado del sistema en el futuro –en el hipotético caso de que aumente el número de municipios, por ejemplo–, o su mejora continua –en el supuesto de que se desee hacer actualizaciones periódicas de los datos para mantener el modelo fiable y actual. Este objetivo abarca realizar una adecuada modularización del proyecto software, desacoplar en la mayor medida posible los datos de sus fuentes, utilizar estructuras de datos agnósticas a la presentación final, y aplicar los principios de responsabilidad única, por ejemplo, junto con emplear las mejores prácticas posibles en la implementación técnica de cada fase, como se describirá posteriormente.
    \item \textbf{Garantizar un adecuado nivel de seguridad jurídica y técnica en cuanto a la puesta en marcha se refiere.} Como parte integral del diseño y desarrollo del sistema, y aplicando los conocimientos de la parte del máster relacionada con las ramas de seguridad y derecho informáticos, se ha pretendido aplicar los principios de seguridad y privacidad por diseño y por defecto, incluidos en el Reglamento Europeo de Protección de Datos~\cite{aepd}, que buscan considerar desde el primer momento de ideación la información personal con la que tratará el sistema, intentando minimizarla en la medida de lo posible y restringirla a la imprescindible para funcionar adecuadamente. Esto es especialmente deseado, tal como aboga el Reglamento, en aquellas actividades que conlleven la creación de perfiles individuales o que procesen cantidades de datos de forma masiva, características que podría tener el sistema resultante de este trabajo.
    \item \textbf{Validar el sistema con usuarios potenciales reales.} Se buscará obtener retroalimentación de personas cercanas al autor que cumplan las condiciones para ser usuarios potenciales del resultado final del trabajo, a fin de validar cualitativamente diversos aspectos del mismo, como la utilidad, capacidad de interacción, tiempo de respuesta del sistema, facilidad de uso, satisfacción con la forma de organizar y presentar la información, coherencia de las respuestas ofrecidas frente a las esperadas, etc. Para ello, se hará uso de diversos conocimientos adquiridos en la asignatura \guillemotleft Visualización de Datos\guillemotright.
    
\end{enumerate}