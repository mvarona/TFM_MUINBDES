\capitulo{1}{Introducción}

Los sistemas de recomendación se han convertido en una de las aplicaciones más usadas dentro del abanico de posibilidades que ofrecen las tecnologías que explotan \textit{Big Data}; es decir, que se aprovechan de una gran cantidad de datos para funcionar. Encontramos estos sistemas de recomendación en redes sociales, portales audiovisuales o cualquier otra plataforma de consumo de contenido –como libros o noticias–, donde nos pueda interesar conocer más sobre alguna temática por la que hemos manifestado un interés previamente. Estos sistemas están, por tanto, especialmente indicados para casos de uso en los que la oferta posible supera con creces la facilidad natural con la que el contenido sería descubierto por los interesados en ausencia del mecanismo de sugerencias automáticas. Si se implementan correctamente, estos mecanismos se convierten en componentes del sistema en los que el usuario confía a la hora de descubrir nuevos grupos musicales, información relacionada o, por qué no, destinos vacacionales o permanentes, atribuyéndoles un argumento de autoridad que en el mundo analógico es habitualmente otorgado a personas de confianza para el usuario, o que considera expertos en la materia.

Partiendo de esa base, y aplicando todos los contenidos vistos en el Máster para el que esta memoria constituye el entregable principal del trabajo final, se ha deseado construir un sistema de recomendación con diversas características que se irán presentando a lo largo de la misma. Pese a que se hace un uso extenso de los conocimientos adquiridos en las asignaturas ``Aprendizaje No Supervisado'' y ``Técnicas de Aprendizaje Automático Escalables'', se ha querido aprovechar el mayor número posible de competencias adquiridas en el resto de materias, con especial mención a las procedentes de ``Modelos de Programación para el Big Data'', ``Infraestructura para el Big Data'', ``Visualización de Datos'', ``Derecho en Seguridad de Datos'' y ``Fundamentos de Ciberseguridad''. En cualquier caso, las aportaciones de cada asignatura se señalarán convenientemente a lo largo de esta memoria.

Con la intención de aunar estos conocimientos en un trabajo teórico-práctico cuyo resultado pueda ser de utilidad, y dentro de la función de transferencia y aplicación de conocimiento que el alumno considera que debe exigirse al término de unos estudios universitarios, se propone como idea el diseño y creación de un sistema de recomendación de municipios de toda España, elegida por los motivos que a continuación se exponen:

\begin{itemize}
    \item \textbf{Necesidad de tratar con una gran cantidad de datos.} Pese a que en un primer momento se barajó la idea de acotar el ámbito a municipios de Castilla y León (2.248 frente a los 8.131 nacionales), y aprovechando la gran cantidad de datos abiertos en formato reusable que esta comunidad ofrece y que facilitaría en gran medida las labores de extracción, transformación, limpieza e ingestión de datos en el sistema, se descartó por diversos motivos: En primer lugar, las labores de recogida y limpieza de datos son tan importantes en un proyecto de Big Data como cualquier otra fase, consumiendo habitualmente entre un 50 \% y un 80 \% de los recursos del proyecto \cite{data_prep}. Esta regla asomaba intuitivamente en asignaturas que requerían algún tipo de ingestión de datos, como ``Arquitecturas Big Data'' y ``Almacenamiento Escalable'', pero no ha sido hasta la elaboración de este trabajo –en el que los datos no se suministraban como parte de un enunciado– cuando el alumno la ha podido comprobar empíricamente. En su caso, y dado que ha procurado usar fuentes de datos abiertas en todo lo posible (de ámbito nacional siempre que han estado disponibles, con la intención de evitar manejar fuentes de 19 comunidades y ciudades autónomas distintas), alrededor de un tercio del tiempo de desarrollo total del trabajo han sido invertidos en esta fase, lo que está estrechamente relacionado con el siguiente punto.
    \item \textbf{Deseo de aplicar técnicas de minería y extracción de datos.} Si bien es cierto que en el máster cursado se ha profundizado en diferentes técnicas para mantener una gran cantidad de datos en infraestructuras adecuadas (que escalen horizontalmente y permitan lecturas y escrituras rápidas), en maneras de obtener conclusiones sobre ellos gracias a técnicas estadísticas y de inteligencia artificial, en formas de mantenerlos seguros a nivel técnico y jurídico, y de visualizarlos cómodamente; las limitaciones lógicas de temario y tiempo han impedido incidir en técnicas de extracción automática de datos. Ha sido gracias a otros recursos aprendidos de forma autodidacta, como el alumno ha conocido técnicas de extracción automática de datos en Internet, cuya puesta en práctica se perfilaba casi obligada en el trabajo final de estos estudios.
    \item \textbf{Intención de crear un sistema de recomendación basado en contenido y en la retroalimentación de los usuarios.} El alumno deseaba poner en práctica sus conocimientos de los dos principales tipos de recomendadores: los basados en contenido y los que utilizan un filtro colaborativo empleando las preferencias de otros usuarios. A lo largo de esta memoria se abordarán ambos y su resultado final.
    \item \textbf{Voluntad de aportar una solución para descubrir ciudades y pueblos españoles.} En numerosas ocasiones se presenta el dilema de qué sitio elegir como destino vacacional o residencial. Hasta ahora, el lugar donde se encontrara trabajo condicionaba enormemente dónde se podría establecer una persona, y los destinos turísticos clásicos ocupaban la mayor parte de la oferta. En la actualidad, la masificación de los mismos ha traído consigo un deseo de explorar nuevas zonas, la mejora de las comunicaciones e infraestructuras ha provocado un resurgimiento en el interés por las zonas rurales, y el coste de vida asociado a las grandes urbes –destino de trabajo obligado para mucha gente– ha provocado la añoranza de una vida más cómoda, fácil de conseguir en una ciudad mediana o pequeña \cite{elpais}, \cite{terrenos}. Por su parte, el incremento del trabajo remoto desde la situación socio-sanitaria derivada de la crisis de la CoViD-19, ha materializado estos factores en un éxodo urbano experimentado en un porcentaje de población lo suficientemente significativo como para ser tenido en cuenta \cite{innovando}, \cite{20mins}. En nuestro país, el caso paradigmático es el de la ciudad de Madrid, que hasta ahora ha aglutinado a personas emigrantes del interior peninsular en busca de oportunidades laborales, pero no es el único. En el sur y archipiélagos peninsulares se ha observado el caso de trabajadores remotos de todo el mundo que han acudido a ellos para realizar sus funciones desde allí, y la tendencia es claramente alcista \cite{expansion}. Desde el ámbito mediático y político se puede apreciar como frecuentemente se iguala el concepto de ``España vaciada'' con territorios muy poco habitados o despoblados \cite{muycomputerpro}, olvidando que, para muchas personas, las limitaciones en municipios así son muy grandes como para plantearse el cambio desde una gran ciudad. Sin embargo, para el autor, esta España despoblada incluye inexorablemente a ciudades, capitales de provincia en muchos casos, que en las últimas décadas han visto disminuir su población drásticamente hacia los sumideros de las grandes áreas urbanas nacionales. Es el caso de la comunidad de las tres universidades de este máster, y es un motivo más para que el alumno no quisiera excluir a ciudades tan válidas para visitar o vivir como Burgos, León, Valladolid o Zamora, por poner algunos ejemplos. Además, y como se detallará en secciones posteriores, el trabajo con las capitales de provincia ha permitido aportar datos muy relevantes, que complementan la información de los municipios cuando, por su tamaño, no es posible disponer de esos datos.
    \item \textbf{Aprovechamiento de experiencia previa con datos abiertos.} El alumno ha querido aprovechar su conocimiento y trabajo previo con diversos catálogos de datos abiertos, como los de la Junta de Castilla y León, los del Gobierno de España o los de la Unión Europea. A mayores, ha querido experimentar con catálogos de datos procedentes de iniciativas privadas o extracción de datos existentes en Internet.
    \item \textbf{Materialización de una idea premiada previamente.} La idea de este trabajo fue una de las seis premiadas el pasado curso en el marco del Programa de Prototipos TCUE (Transferencia de Conocimiento Universidad-Empresa) de la Fundación Universidad de Burgos, dirigido a potenciar la creación de prototipos a partir de trabajos de fin de grado, máster o tesis doctorales que puedan aportar algo novedoso al mercado \cite{ubu}. Por otro lado, también fue premiada con el segundo premio en la categoría ``Ideas'' del V Concurso de Datos Abiertos de la Junta de Castilla y León \cite{cyl}, por lo que al recibir esta validación externa, se consideró que se debía intentar llevar a cabo. 
    \item \textbf{Deseo de aportar una solución novedosa.} El alumno considera que, además de demostrar y aplicar los conocimientos adquiridos a lo largo de los estudios universitarios, los trabajos finales de grado y máster también deben poder ser publicables; en el sentido de poder devolver a la sociedad parte de lo adquirido gracias a una formación pública de calidad que ella misma ha sufragado. Por ello, se ha diseñado también una forma de distribución de los resultados de este trabajo adecuada al público objetivo, como se detallará posteriormente. Asimismo, si bien se ha encontrado un servicio similar para filtrar por diversos criterios los municipios de Castilla y León \cite{repueblame}, o diversas iniciativas para descubrir pueblos de la España vaciada \cite{apadrina_un_pueblo}, \cite{pueblos_magicos}, no se ha encontrado ninguna alternativa para descubrir ciudades y pueblos españoles, sin importar su densidad de población, y cuya recomendación esté personalizada para cada usuario.
 \end{itemize}
